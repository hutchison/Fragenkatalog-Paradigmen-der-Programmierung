\documentclass[
  a4paper,
  11pt,
]{article}

\usepackage[utf8]{inputenc}
\usepackage[cm, headings]{fullpage}
\usepackage[ngerman]{babel}
\usepackage{amsmath}
\usepackage{amssymb}
\renewcommand{\O}{\mathcal{O}}

\usepackage{color}
\definecolor{mygray}{rgb}{0.5,0.5,0.5}
\usepackage{listings}
\usepackage{tikz}
\usepackage{pgfplots}

% Für Zeilenumbrüche ohne Indentations
\setlength{\parindent}{0pt}

\lstset{%
  basicstyle=\ttfamily,
  keywordstyle=\color{blue},
  commentstyle=\color{mygray},
  language=C,
  showstringspaces=false,
}

% coole Kopf- und Fußzeilen:
\usepackage{fancyhdr}
% Seitenstil ist natürlich fancy:
\pagestyle{fancy}
% alle Felder löschen:
\fancyhf{}

\fancyhead[L]{%
  Paradigmen der Programmierung
}
\fancyhead[R]{%
  Fragenkatalog
}
%\fancyfoot[L]{}
\fancyfoot[C]{\thepage}

\newcommand{\gge}{>\!\!>\!\!=}

\title{}

\author{}

\begin{document}

\thispagestyle{fancy}

\subsection*{Unix}
\label{sub:Unix}

\begin{itemize}
  \item Beschreiben Sie ein allgemeines Unixsystem.
  \item Beschreiben Sie die Philosophien von Unix.
  \item Beschreiben Sie das Prinzip „alles ist eine Datei.“
    \begin{itemize}
      \item Warum ist das sinnvoll?
      \item Welches Paradigma wird dadurch implementiert?
        (Pipe-Filter-Architektur bzw. Datenflussparadigma)
    \end{itemize}
  \item Wo liegen in Unix/Linux die Gerätetreiber? In den Programmen oder im
    Kernel? (Kernel)
  \item Wenn man ein neues Gerät an den Computer anschließt, müssen Sie das dann
    in ihrer Anwendung berücksichtigen?
  \item Wenn ein C-Programm eine Minute läuft, kann man dann sicher sein, dass
    dann immer noch ein C-Programm läuft? Besonders im Hinblick auf den
    \texttt{exec}-Systemcall.
  \item Erklären Sie den Unterschied zwischen \texttt{fork()} und
    \texttt{exec()}.
  \item Was passiert genau bei \textt{fork()} und \texttt{exec()}?
  \item Wenn ein Programm mittels \texttt{fork()} ein Kindprogramm erzeugt, ab
    welchem Punkt findet dann die weitere Ausführung statt? (ab dem Aufruf von
    \texttt{fork()} und nicht von vorne)
  \item Wie baut man Pipes? Wie funktionieren Pipes?
  \item Was passiert beim Umleiten der Ausgabe eines Programms? Wie genau wird
    das in einer Shell gemacht?
  \item Warum sind \textit{file descriptor} wichtig?
  \item Welche Standard-FDs bekommt jeder Prozess? (\texttt{stdin},
    \texttt{stdout}, \texttt{stderr})
\end{itemize}

\subsection*{Haskell und Monaden}
\label{sub:Haskell und Monaden}

\begin{itemize}
  \item Was ist eine Monade?
  \item Wofür braucht man Monaden?
  \item Betrachten wir die IO-Monade aus der Vorlesung: welchen Typ hat diese
    und wie ist der Bind-Operator ($\gge$) definiert? (auf „Weltzustand“
    und „RealWorld“ eingehen)
  \item Wie funktioniert dann die sequentelle Berechnung/Ausführung von
    Funktionen?
  \item Was ist ein Monadentransformer?
  \item Wie sieht die Typklasse von Monadentransformern aus?
  \item Wie schafft man es einen Zustand (State) an eine andere Monade
    ranzubacken?
\end{itemize}

\subsection*{Lisp}
\label{sub:Lisp}

\begin{itemize}
  \item Was ist so besonders an Lisp?
  \item Nennen Sie Unterschiede zwischen Lisp und Haskell.
  \item Was ist ein Makro in Lisp?
  \item Wie kann man ein Lisp-Programm mit Lisp verändern?
  \item Was ist eine Closure?
  \item Erklären Sie den Unterschied zwischen einem C- und einem Lisp-Makro.
  \item Erklären Sie den Unterschied zwischen einer Haskell- und einer
    Lisp-Closure.
\end{itemize}

\end{document}
